\documentclass[12pt,a4paper]{article}

\usepackage{ctex}
\usepackage{amsmath}

\title{微分方程与差分模型}
\author{longlin}
\date{\today}

\begin{document}
\maketitle

\section{微分方程的求解}
\subsection{符号解与数值解}
\begin{itemize}
    \item 符号解:完整写出解的代数式,强调解的代数性
    \item 数值解:算出具体数值,不需完完全全精确,满足一定要求即可
\end{itemize}

\subsection{多元函数常/偏微分方程组}
\begin{itemize}
    \item 偏微分方程和初始条件组成的问题叫做初值问题(Cauchy问题)
    \item 偏微分方程和第一类边界条件组成的问题叫做第一类边值问题(Dirichlet问题)
    \item 偏微分方程和第二类边界条件组成的问题叫做第二类边值问题(Neumann问题)
    \item 偏微分方程和第三类边界条件组成的问题叫做第三类边值问题(Robin问题)
    \item 既有初始条件又有边界条件的问题叫做混合问题
\end{itemize}

\section{微分方程案例}
\subsection{人口增长模型}
\subsubsection{Malthus模型}
马尔萨斯模型假设增长率永远是个常数r,那么一段时间内增长的个体有
$$
x(t + \Delta t) - x(t) = rx(t)\Delta t
$$

按照微分方程的形式整理:
$$
\begin{cases}
    \dfrac{\textrm{d} x}{\textrm{d} t} = rx \\
    x(0) = x_0
\end{cases}
$$

解得:
$$
x(t) = x_0 e^{rt}
$$

\subsubsection{Logistic模型}
现在对模型进行一些修正,假设增长率会随着种群数量增加而衰减,种群最大的一个平衡数量叫K,那么可以得到
$$
\begin{cases}
    \dfrac{\textrm{d} x}{\textrm{d} t} = rx(1 - \dfrac{x}{K}) \\
    x(0) = x_0
\end{cases}
$$
$$
\text{解得}
\quad 
x(t) = \dfrac{K}{1+\left ( \frac{K}{x_0} - 1 \right ) e^{-rt}}
$$

\end{document}