\documentclass[12pt,a4paper]{article}

\usepackage{ctex}
\usepackage{enumerate}
\usepackage{amsmath}

\title{线性规划模型笔记}
\author{longlin}
\date{\today}

\begin{document}
\maketitle

\section{线性规划模型}
\subsection{线性规划的矩阵形式}
\begin{itemize}
    \item 不等式组条件矩阵化
    \item 方程组条件矩阵化
    \item 每个变量自己的取值范围
    \item 目标函数的向量化
    \item 求极值
\end{itemize}

$$
\min_x \: c^T X \\
$$
\begin{equation}
    s.t. \:
    \begin{cases}
        & Ax \leq b \\
        & A_{eq} \cdot x = b_{eq} \\
        & lb \leq x \leq ub
    \end{cases}
\end{equation}
如果遇到求极大值问题,可以将目标函数取负号,转化为求极小值问题。
如果不等式组中有>=的不等式,可以添负号将其转化为<=的不等式。

\subsection{单纯形法}
国定变量,不断变换基向量求方程组的解带入,看是不是最优解,不是就更新迭代现阶段的解。

单纯形法要求约束条件都为等式,且要所有变量非负。
我们需要把不等式约束$\textbf{通过引入松弛变量变为等式约束}$。

\\
什么情况下要在线性规划中映入松弛变量呢?
\begin{itemize}
    \item 需要将线性规划化成标准形式时
    \item 遇到绝对值问题时
    \item 不等式过多甚至存在非线性的不等关系时
\end{itemize}
\subsection{蒙特卡洛法}
在$\textbf{可行域范围内生成大批量随机数据点}$,观测这些数据点在什么位置取得近似最优。
如此得到近似最优解,更多的适用于求解非线性问题。

\end{document}