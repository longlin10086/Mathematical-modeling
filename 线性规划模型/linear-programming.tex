\documentclass[12pt,a4paper]{article}

\usepackage{ctex}
\usepackage{enumerate}
\usepackage{amsmath}

\title{线性规划模型笔记}
\author{longlin}
\date{\today}

\begin{document}
\maketitle

\section{线性规划模型}
\subsection{线性规划的矩阵形式}
\begin{itemize}
    \item 不等式组条件矩阵化
    \item 方程组条件矩阵化
    \item 每个变量自己的取值范围
    \item 目标函数的向量化
    \item 求极值
\end{itemize}

$$
\min_x \: c^T X \\
$$
\begin{equation}
    s.t. \:
    \begin{cases}
        & Ax \leq b \\
        & A_{eq} \cdot x = b_{eq} \\
        & lb \leq x \leq ub
    \end{cases}
\end{equation}
如果遇到求极大值问题,可以将目标函数取负号,转化为求极小值问题。
如果不等式组中有>=的不等式,可以添负号将其转化为<=的不等式。

\subsection{单纯形法}
国定变量,不断变换基向量求方程组的解带入,看是不是最优解,不是就更新迭代现阶段的解。

单纯形法要求约束条件都为等式,且要所有变量非负。
我们需要把不等式约束$\textbf{通过引入松弛变量变为等式约束}$。

什么情况下要在线性规划中映入松弛变量呢?
\begin{itemize}
    \item 需要将线性规划化成标准形式时
    \item 遇到绝对值问题时
    \item 不等式过多甚至存在非线性的不等关系时
\end{itemize}
\subsection{蒙特卡洛法}
在$\textbf{可行域范围内生成大批量随机数据点}$,观测这些数据点在什么位置取得近似最优。
如此得到近似最优解,更多的适用于求解非线性问题。

\subsection{0-1规划问题}
0-1规划问题是线性规划问题的一种特殊形式,即变量只能取0或1。
\subsubsection{指派问题}
假设n个人恰好做n项任务,第i个人做第j项任务的效率为$c_{ij} \geq 0$,应指派哪个人完成哪项任务,使完成效率最高。

$$
\text{决策变量:} x_{ij} = 
\begin{cases}
    1, & \text{指派第i个人做第j项任务} \\
    0, & \text{不指派第i个人做第j项任务}
\end{cases}
$$

$$
\text{目标函数:} \min Z = \sum_{i=1}^n \sum_{j=1}^n c_{ij} x_{ij}
$$

$$
\text{约束条件:}
\begin{cases}
    \sum\limits_{i=1}^n x_{ij} = 1, & j = 1,2,\cdots,n \\
    \sum\limits_{j=1}^n x_{ij} = 1, & i = 1,2,\cdots,n \\
    x_{ij} \in \{0,1\}, & i,j = 1,2,\cdots,n
\end{cases}
$$

\subsection{动态规划模型}
\textbf{本质上是进行空间换时间的操作}
\begin{enumerate}[1)]
    \item 核心思想:将大问题划分为小问题进行解决,从而一步步获取最优解的处理算法
    \item 与分治法不同的是,动态规划分解得到的子问题往往不是互相独立的。(即下一个子阶段的求解是建立在上一个子阶段的解的基础上,进行进一步的求解)
    \item 动态规划通过填状态表的方式逐步推进,得到最优解。
\end{enumerate}


\textbf{动态规划程序编写思路(k表示多轮决策的第k轮):}
\begin{enumerate}[1.]
    \item 将问题分解成几个子问题,一个子问题就是多轮决策的一个阶段
    \item 选择合适的状态变量S\(k\),它需要具备描述多轮决策过程的演变。
    \item 确定决策变量u\(k\),每一轮的决策就是每一轮可能的决策动作,即当前可以有哪些决策可以选择。
    \item 列写状态转移方程,即所谓递推公式。
    \item 写出代表多轮决策目标的指标函数V\(k,n\),即最终需要达到的目标。
    \item 寻找目标的终止条件。
\end{enumerate}
\end{document}